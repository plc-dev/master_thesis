% \chapter{Theoretical Foundation of Formalizing, Curating and Formulating Commonsense Knowledge}

\chapter{Aspects of Relational Databases, Natural Language Processing and Knowledge Graphs}
\label{ch:theory}

\section{Fundamental Aspects of Relational Databases Regarding the Generation of SQL-Query Exercises}
\label{sec:theory:sql}

\subsection{Related Work in Generating and Marking SQL-Query Exercises}
\label{sec:theory:sql:related}
% Query generation that includes keyword, explain query, but no translation of query present \cite{Do2014AutomaticGO}


% Reverse SQL Question Generation Algorithm (RSQLG), template based generic nlg
% no JOIN, no GROUP BY, checks for exact same querry \cite{Atchariyachanvanich2019ReverseSQ}


% based on \cite{Atchariyachanvanich2019ReverseSQ}, low quality, no joins, no group by, local c app with no binary \cite{Chaudhari2021StructuralQL}


% based on \cite{Atchariyachanvanich2019ReverseSQ}, added DDL \cite{Dwivedi2020AUTOMATICSQ}



% based on context free grammar, theoretical paper, does not actually generate queries \cite{Gudivada2017AutomatedGO}



% identical framework, allows for up to 3 joins, that are pre configured, direct comparison of generated query to user query \cite{10.1145/3436756.3437037, Basse2019OntologyBasedSF}


\subsection{Assessing the Complexity of SQL-Queries}
\label{sec:theory:sql:complexity}
\textit{Complexity} is an ambiguous concept whose concrete definition is dependent on the context domain. Complexity science can be seen as the \textit{study of the phenomena which emerge from a collection of interacting objects} \cite{johnson2007two}. Applying this definition to SQL-queries, the complexity of a generated SQL-query can be defined as being governed by the following three criteria:
\begin{enumerate}
 \item The number of distinct SQL-keywords that occur in the generated SQL-query.
 \item The number of occurrences of a specific SQL-keyword in a generated SQL-query.
 \item The number of arguments a specific SQL-keyword receives.
 \end{enumerate}
 
 
 \cite{Subali2018ANM}
 
 not evaluating effect of database complexity on sql query formulation \cite{Taipalus2020TheEO}

\subsection{Assessing the Complexity of SQL-Queries}
\label{sec:theory:sql:related}

\section{Natural Language Processing for Transforming SQL to Natural Language}
\label{sec:nlp}

\subsection{State-of-the-Art-Analysis of SQL-to-Text Models}
\label{sec:nlp:sota}
\input{chapters/theory/nlp/sota}

\subsection{Uses of the Transformer Architecture in Pretrained Large Language Models}
\label{sec:nlp:transformer}
\input{chapters/theory/nlp/transformer}

\subsection{Natural Language Generation Pipeline}
\label{sec:nlp:transformer}
\input{chapters/theory/nlp/transformer}

\subsection{Applying Pretrained Large Language Models as Surface Realization Engines}

\section{Commonsense Knowledge Graphs as a Foundation for Semantically Enriched Relational Databases}

\subsection{Current Landscape of Commonsense Knowledge Graphs}

\subsection{Entities and Concepts of the Knowledge Graph as Tables}

\subsection{Ontological Relations between Entities and Concepts as Foreign Key Constraints and Junction Tables}

\subsection{Entity and Concept Instances of the Knowledge Graph as Table Records}

% \subsection{Development and Lifecycle of Large Language Models}

% \subsection{Ethical Concerns of Large Language Models}


% \section{Crowdsourcing for Supervision of Knowledge Generation}

% \subsection{Applications of Crowdsourcing in Artificial Intelligence}

% \subsection{Survey Design Principles}

% \subsection{Ethical Concerns of Crowdsourcing}


% PUT INTO NLP CHAPTER IN THE BEGINNING
% \textit{Selectional preference} refers to how typical an event or situation is perceived, measured by its observed frequency. As humans avoid stating the obvious  \textit{} \cite{Wang2018ModelingSP}