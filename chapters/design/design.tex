\chapter{Design and Implementation of a System for the Generation of Meaningful SQL-Query Exercises}
\label{ch:des}

\section{Overview of a System for the Generation of Meaningful SQL-Query Exercises}
\label{sec:des:overview}
\input{chapters/design/sections/overview}

\section{Deriving a Relational Database from a Knowledge Graph}
\label{sec:des:semantic}

\subsection{Deriving a Relational Schema from Entities and Concepts of the Knowledge Graph}
\label{sec:des:schema}
wikipedia categories
    extract 
    ignore cycle

\subsection{Inferring Cardinalities from the Entity and Concept Instances of the Knowledge Graph}
\label{sec:des:cardinaltiy}

\subsection{Ingesting Entity- and Concept-Instances into the Relational Database}
\label{sec:des:ingestion}

\subsection{Limitations of Commonsense Knowledge Graphs Regarding the Semantic Enrichment of Relational Databases}
\label{sec:des:limits}

\section{Semantically Enriching a Relational Database with a Knowledge Graph}
\label{sec:des:semantic}

\subsection{Normalizing the Relational Database to Satisfy Domain Constraints and Cardinality Restrictions}
\label{sec:des:normal}

\subsection{Knowledge Graph Entities for Semantic Labeling of Tables}
\label{sec:des:entities}

\subsection{Mapping Tables and Foreign Key Constraints of a Relational Schema to Knowledge Graph Entities}
\label{sec:des:mapping}

\subsection{Resolving Implicit Information by Imposing Directionality of Foreign Key Constraints and Junction Tables}
\label{sec:des:resolving}



\section{Generation of Meaningful SQL-Query Exercises}
\label{sec:des:generation}

\subsection{Parameter Space of the Generation Algorithm}
\label{sec:des:parameters}
In order to formulate


\subsection{Traversing the Relational Schema as a Graph of Tables and Foreign Key Constraints}
\label{sec:des:traversal}

\subsection{Generation of Random Parameter-Compliant SQL-Queries}
\label{sec:des:algorithm}

\subsection{Construction of Solution Assessment Functions}
\label{sec:des:assessment}


\section{Generation of an SQL-Query Exercise Formulation in Natural Language}
\label{sec:des:formulation}

\subsection{Extracting the Sentence Content by Assigning Semantic Labels to the SQL-Query Constituents}
\label{sec:des:planning}

\subsection{Deriving the Sentence Structure from the SQL-Query Structure}
\label{sec:des:structuring}

\subsection{Completing the Structured Sentence}
\label{sec:des:surface}



% \section{Task interaction}
% \label{sec:des:interaction}

% \subsection{Visual task arrangement}
% \label{sec:des:arrangement}

% \subsection{Interaction recording and replay}
% \label{sec:des:record}