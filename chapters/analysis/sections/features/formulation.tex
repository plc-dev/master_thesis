For each generated SQL-query, the system shall formulate a description in human-like natural language of the corresponding result set.

The exercise description shall consist of a depiction of the underlying database schema and a description of the result set.

What makes machine-generated text \textit{human-like} differs depending on the domain and task and the task-dependent evaluation of this property often lacks concrete specifications \cite{Lee2021HumanEO, VANDERLEE2021101151}. Thus for the particular task of generating descriptions of SQL-queries, the following criteria determine the property of human-like:
\begin{enumerate}

 \item Conciseness
 \begin{enumerate}
    \item Length of text, for which shorter is better.
    \item Information aggregation, for which less explicit phrasing is better.
 \end{enumerate}
 
 \item Correctness
 \begin{enumerate}
    \item Missing information, where less is better.
    \item Spelling mistakes, where less are better.
    \item Grammar mistakes, where less are better.
 \end{enumerate}
 
 \item Style
 
\end{enumerate}


% that describes the task of replicating a query whose execution leads to the same result set

